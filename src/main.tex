\documentclass[a5paper,12pt]{book}

\usepackage[
    a5paper,
    scale = 2.4,
    left = 2cm,
    right = 2cm,
    top = 2cm,
    bottom = 2.7cm
] {geometry}



% 颜色
\usepackage[svgnames]{xcolor}
\definecolor{textgray}{rgb}{0.25,0.25,0.25}
\definecolor{by}{rgb}{0.3,0.4,0.5}
\color{textgray}


% 章节格式
\usepackage{titlesec}
\titleformat{\chapter}{\centering\Huge\bfseries}{章\,\thechapter}{1em}{}
\titleformat{\section}{\bf\LARGE}{x,\section\,y}{2em}{}

% 目录格式
\renewcommand{\contentsname}{\textbf{目录}}

% 脚注格式
\usepackage[perpage]{footmisc}
\renewcommand*{\thefootnote}{\textcolor{by}{〔\arabic{footnote}〕}}
\setlength{\footnotemargin}{8mm}
\setlength{\footnotesep}{\baselineskip}

% 水印及格式
\usepackage[
    CJKbookmarks = true,
    bookmarks = true,
    linktocpage = true,
    hyperindex = true,
    bookmarks = true,
    pdfborderstyle={/S/U/W 0.5}% border style will be underline of width 1pt
] {hyperref}
\newcommand{\htoday}{
   \href{https://github.com/DingShizhe}{\today}
}
\usepackage{draftwatermark}
\SetWatermarkText{\htoday}
\SetWatermarkLightness{0.97}


% 页面格式
\usepackage{fancyhdr}

\renewcommand{\headrulewidth}{0.0pt}
\renewcommand{\footrulewidth}{0.0pt}

\fancyfoot[CO,RE]{}
\pagestyle{fancy}
\fancyhf{}
\fancyhead[ER]{\leftmark}
\fancyhead[OL]{\rightmark}
\fancyhead[EL,OR]{\thepage}
\renewcommand{\headrulewidth}{0.4pt}
\fancypagestyle{plain}{%
    \fancyhf{}
    \fancyfoot[C]{\bfseries \thepage}
    \renewcommand{\headrulewidth}{0pt}
    \renewcommand{\footrulewidth}{0pt}
}


% 字体
\usepackage{xeCJK}
\setCJKmainfont[
    ItalicFont=华文楷体,
    BoldFont=方正小标宋_GBK
]{华文中宋}

\setCJKfallbackfamilyfont{\CJKfamilydefault}{
    [ItalicFont=华文楷体]{宋体}
}

\newCJKfontfamily\FangZhengSongYi{方正宋一_GBK}         % ok
\newCJKfontfamily\FangZhengBaoSong{方正报宋_GBK}        % ok
\newCJKfontfamily\FangZhengSongSan{方正宋三_GBK}        % 缺繁体
\newCJKfontfamily\SongTi{宋体}                          % ok
\newCJKfontfamily\XinSongTi{新宋体}                     % ok
\newCJKfontfamily\HuaWenSongTi{华文宋体}                % ok
\newCJKfontfamily\FangZhengShuSong{方正书宋_GBK}        % ok

\newCJKfontfamily\FangZhengDaBiaoSong{方正大标宋_GBK}   % 缺字
\newCJKfontfamily\FangZhengXiaoBiaoSong{方正小标宋_GBK} % ok
\newCJKfontfamily\FangZhengCuSong{方正粗宋_GBK}         % 缺字
\newCJKfontfamily\HuaWenZhongSong{华文中宋}             % ok
\newCJKfontfamily\FangZhengFengYaSong{方正风雅宋简体}    % 缺繁体
\newCJKfontfamily\FangZhengSongHei{方正宋黑_GBK}        % 缺繁体

\newCJKfontfamily\HuaWenFangSong{华文仿宋}              % ok
\newCJKfontfamily\FangZhengFangSong{方正仿宋_GBK}       % ok

\newCJKfontfamily\HuaWenKaiTi{华文楷体}                 % ok
\newCJKfontfamily\HanYiKaiTi{汉仪楷体S}
\newCJKfontfamily\FangZhengKaiTi{方正楷体_GBK}

\newCJKfontfamily\FandolKai{FandolKai}
\newCJKfontfamily\FandolFang{FandolFang}
\newCJKfontfamily\FandolSong{FandolSong}
\newCJKfontfamily\FandolHei{FandolHei}
% \newCJKfontfamily\FandolSongBold{FandolSong}
% \newCJKfontfamily\FandolHeiBold{FandolHei}


\usepackage{bbding}  % symbols
\usepackage{fourier-orns} %symbols

\setcounter{secnumdepth}{0}

\begin{document}

    \setlength{\parindent}{0pt}
    \setlength{\baselineskip}{20pt}
    \pagestyle{plain}

    
\newcommand*{\plogo}{\FiveFlowerOpen}

\newcommand{\Author}{\href{https://github.com/DingShizhe}{DingShizhe}}
\newcommand\myquotepage[3]{\thispagestyle{empty}\vspace*{\fill}\vspace*{\fill}\textcolor{#1}{\textit{#2\\\rightline{#3}}}\vspace*{\fill}\vspace*{\fill}\vspace*{\fill}}

\begin{titlepage}
	\raggedleft
	\rule{1pt}{\textheight} % vertical line
	\hspace{0.06\textwidth}
	\parbox[b]{0.80\textwidth}{
		{
            {\fontsize{40}{50}\selectfont
            \it 拾穗}
        }\\[2\baselineskip] % Title
		\textcolor{Sienna}{\large\textit{附庸风雅而已}}\\[4\baselineskip]{\large\textsc{\Author}}
        
		\vspace{0.5\textheight}
		\color{Sienna}{\noindent 且慢~\plogo}\\[\baselineskip]
	}
    
\end{titlepage}

    \input{./句/16-有匪君子,如切如磋,如琢如磨。.tex}

    \tableofcontents

    \setcounter{page}{1}

\chapter{诗}
    \newpage

    \input{./句/2-蒹葭苍苍,白露为霜。所谓伊人,在水一方。.tex}
    \input{./诗/佚名-涉江采芙蓉.tex}
    \input{./诗/佚名-陌上桑.tex}
    \input{./诗/佚名-江南.tex}
    \input{./诗/佚名-氓.tex}
    \input{./诗/佚名-苕之华.tex}

    \input{./诗/佚名-悲歌.tex}

    \input{./诗/王维-送别.tex}
    \input{./诗/常建-题破山寺后禅院.tex}
    \input{./句/34-万里赴戎机,关山度若飞。.tex}
    \input{./诗/王维-使至塞上.tex}

    \input{./诗/李白-春思.tex}
    \input{./诗/李商隐-晚晴.tex}
    
    \input{./句/3-生年不满百,常怀千岁忧。昼短苦夜长,何不秉烛游?.tex}
    
    \input{./诗/王绩-野望.tex}
    \input{./诗/柳宗元-江雪.tex}
    \input{./诗/钱起-省试湘灵鼓瑟.tex}
    \input{./诗/李商隐-夜雨寄北.tex}
    \input{./诗/王勃-送杜少府之任蜀州.tex}
    \input{./诗/杜甫-望岳.tex}
    \input{./诗/杜甫-春望.tex}

    \input{./诗/李商隐-乐游原.tex}
    \input{./诗/李白-送友人.tex}
    \input{./句/9-人生天地间,忽如远行客。.tex}
    \input{./诗/杜甫-旅夜书怀.tex}
    \input{./诗/杜甫-春夜喜雨.tex}

    \input{./句/5-岁寒,然后知松柏之后凋也。.tex}
    
    \input{./句/7-人生不相见,动如参与商。今夕复何夕,共此灯烛光。.tex}
    

    \input{./诗/王维-杂诗三首其二.tex}

    \input{./诗/陈子昂-登幽州台歌.tex}

    \input{./诗/张九龄-望月怀远.tex}

    \input{./诗/张九龄-感遇其一.tex}

    \input{./诗/王维-相思.tex}
    \input{./诗/祖咏-终南望余雪.tex}

    \input{./句/10-我们自古以来,就有埋头苦干的人,有拼命硬干的人.tex}

    \input{./诗/王翰-凉州词二首·其一.tex}

    % \input{./句/11-在秋风鲈鱼之前,尚有薄酒一樽。.tex}

    \input{./诗/宋之问-渡汉江.tex}
    \input{./诗/白居易-问刘十九.tex}
    
    \input{./句/12-即便我身处果壳之中,仍自以为是无限宇宙之王。.tex}

    \input{./诗/梅尧臣-陶者.tex}
    \input{./句/13-长太息以掩涕兮,哀民生之多艰!.tex}

    \input{./诗/李绅-悯农二首.tex}

    \input{./句/25-每个人对于他所属于的社会都负有责任,那个社会的弊他也有一份。.tex}
    \input{./诗/张俞-蚕妇.tex}

    \input{./句/17-我一直很偏执地记录这些人,甚至到了他们自己都忘记自己的时候。.tex}

    \input{./句/14-滚滚长江东逝水,浪花淘尽英雄。.tex}

    \input{./句/20-湮没了黄尘古道,荒芜了烽火边城。.tex}

    \input{./句/16-有匪君子,如切如磋,如琢如磨。.tex}
    
    \input{./诗/李白-赠从兄襄阳少府皓.tex}

    \input{./句/19-子在川上曰:逝者如斯夫。.tex}

    \input{./诗/李白-梦游天姥吟留别.tex}

    \input{./句/21-三军可夺帅也,匹夫不可夺志也。.tex}

    \input{./诗/王维-竹里馆.tex}

    \input{./诗/黄庭坚-寄黄几复.tex}

    \input{./诗/苏轼-和子由渑池怀旧.tex}

    \input{./诗/刘禹锡-酬乐天扬州初逢席上见赠.tex}
    \input{./诗/陆游-十一月四日风雨大作.tex}
    \input{./诗/杜甫-客至.tex}
    \input{./诗/杜甫-登高.tex}
    \input{./诗/李白-渡荆门送别.tex}
    \input{./诗/韩愈-左迁至蓝关示侄孙湘.tex}
    \input{./诗/陆游-书愤五首·其一.tex}


\chapter{词曲}
    \newpage
    \input{./句/8-一弹戏牡丹,一挥万重山。.tex}

    \input{./诗/李白-忆秦娥·箫声咽.tex}
    \input{./诗/李白-菩萨蛮·平林漠漠烟如织.tex}
    \input{./诗/溫庭筠-望江南·梳洗罷.tex}
    \input{./诗/张先-天仙子·水调数声持酒听.tex}
    \input{./诗/苏轼-浣溪沙·细雨斜风作晓寒.tex}
    \input{./诗/苏轼-念奴娇·赤壁怀古.tex}
    \input{./诗/苏轼-蝶恋花·春景.tex}
    \input{./诗/秦观-鹊桥仙·纤云弄巧.tex}
    \input{./诗/秦观-踏莎行·郴州旅舍.tex}
    \input{./诗/苏轼-水调歌头·明月几时有.tex}
    \input{./诗/柳永-雨霖铃·寒蝉凄切.tex}
    \input{./诗/歐陽修-長相思·花似伊.tex}
    \input{./诗/周邦彦-少年游·并刀如水.tex}
    \input{./诗/李清照-点绛唇·蹴罢秋千.tex}
    \input{./诗/李清照-如梦令·常记溪亭日暮.tex}
    \input{./诗/李清照-如梦令·昨夜雨疏风骤.tex}
    \input{./诗/冯延巳-谒金门·风乍起.tex}
    \input{./诗/柳永-蝶恋花·伫立危楼.tex}
    \input{./诗/李煜-長相思·一重山.tex}
    \input{./诗/岳飞-小重山·昨夜寒蛩不住鸣.tex}
    \input{./诗/李清照-声声慢·寻寻觅觅.tex}
    \input{./诗/王觀-卜算子·送鲍浩然之浙东.tex}
    \input{./诗/柳永-八声甘州·对潇潇暮雨洒江天.tex}
    \input{./诗/辛弃疾-青玉案·元夕.tex}
    \input{./诗/辛弃疾-西江月·夜行黄沙道中.tex}
    \input{./诗/纳兰性德-山一程·长相思.tex}
    \input{./诗/周邦彦-苏幕遮·燎沉香.tex}
    \input{./诗/苏轼-望江南·超然台作.tex}
    \input{./诗/苏轼-卜算子·黄州定慧院寓作.tex}
    \input{./诗/苏轼-定风波·三月七日.tex}
    \input{./诗/苏轼-临江仙·夜饮东坡醉复醒.tex}
    \input{./诗/苏轼-定风波·南海归赠王定国侍人寓娘.tex}
    \input{./诗/晏殊-蝶恋花·槛菊愁烟兰泣露.tex}
    \input{./诗/晏殊-浣溪沙·一曲新词酒一杯.tex}
    \input{./诗/晏几道-临江仙·梦后楼台高锁.tex}
    \input{./诗/姜夔-扬州慢·淮左名都.tex}
    \input{./诗/陆游-钗头凤·红酥手.tex}
    \input{./诗/陆游-诉衷情·当年万里觅封侯.tex}
    \input{./诗/邱圆-寄生草·漫揾英雄泪.tex}

\chapter{现代诗}
    \newpage
    \input{./句/15-我和世界皆不完美。.tex}
    \input{./诗/毕赣-2016.1.5.tex}

    \input{./诗/余秀华-风从田野上吹过.tex}
    \input{./诗/张枣-镜中.tex}
    \input{./诗/余秀华-写给门前的一棵树.tex}

    \input{./诗/李叔同-送别.tex}
    \input{./句/6-问余何适,廓尔忘言,华枝春满,天心月圆。.tex}
    \input{./诗/废名-掐花.tex}
    \input{./诗/北岛-真的.tex}
    \input{./诗/余秀华-穿越大半个中国去睡你.tex}
    \input{./诗/郑愁予-错误.tex}
    \input{./诗/余秀华-我爱你.tex}
    \input{./诗/废名-十二月十九夜.tex}

    \input{./句/23-酒入愁肠,七分酿成月光。余下的三分呼为剑气,绣口一吐便是半个盛唐。.tex}

    \input{./诗/余光中-飞将军.tex}

\chapter{文}


    \newpage


    \input{./句/22-臣本布衣,躬耕于南阳。.tex}

    \input{./文/陶渊明-桃花源记.tex}

    \input{./文/吴均-与朱元思书.tex}

    \input{./文/王羲之-兰亭集序.tex}

    \input{./文/郦道元-三峡.tex}

    \input{./句/26-天下兴亡,匹夫有责。.tex}

    \input{./文/萧铎-采莲赋.tex}

    \input{./文/曹植-洛神赋.tex}

    % \input{./文/石崇-金谷诗序.tex}

    \input{./文/李密-陳情表.tex}

    \input{./文/李白-春夜宴从弟桃花园序.tex}

    \input{./文/欧阳修-醉翁亭记.tex}

    \input{./文/周敦颐-爱莲说.tex}

    \input{./句/20-人固有一死,或重于泰山,或轻于鸿毛。.tex}

    \input{./文/柳宗元-小石潭记.tex}

    \input{./句/18-孔子曰:何陋之有?.tex}

    \input{./句/24-燕雀安知鸿鹄之志哉!.tex}

    \input{./句/4-若士必怒,伏尸二人,流血五步,天下缟素。.tex}

    \input{./文/苏轼-前赤壁赋.tex}

    \input{./文/张岱-湖心亭看雪.tex}

    \input{./文/歸有光-項脊軒志.tex}


\end{document}
