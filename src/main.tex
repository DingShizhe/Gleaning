\documentclass{book}

\usepackage[
    a5paper,
    scale = 1,
    left = 2cm,
    right = 2cm,
    top = 2.5cm,
    bottom = 2.8cm
] {geometry}

\usepackage{xeCJK}
\setCJKmainfont[
    % BoldFont={FandolHei},
    BoldFont=方正大标宋_GBK
    % BoldFont=Microsoft YaHei,
    % BoldFont=方正黑体_GBK,
    % BoldFont=方正报宋_GBK,
    % ]{方正楷体_GBK}
% ]{DFKai-SB}
% ]{HYKaiTiS}
% ]{方正宋一_GBK}
% ]{方正宋三_GBK}
% ]{方正小标宋_GBK}
]{华文中宋}


\usepackage{titlesec}
\titleformat{\chapter}{\centering\Huge\bfseries}{章\,\thechapter}{1em}{}
\titleformat{\section}{\bf\LARGE}{x,\section\,y}{2em}{}


\usepackage[
    CJKbookmarks = true,
    bookmarks = true,
    linktocpage = true,
    hyperindex = true,
    bookmarks = true
] {hyperref}

\hypersetup{%
    pdfborderstyle={/S/U/W 0.5}% border style will be underline of width 1pt
}

\usepackage[svgnames]{xcolor}
\definecolor{textgray}{rgb}{0.25,0.25,0.25}
\color{textgray}


%\newcommand{\htoday}{
%    \href{https://github.com/DingShizhe}{\today}
%}
%\usepackage{draftwatermark}
%\SetWatermarkText{\htoday}
%\SetWatermarkLightness{0.97}




\newcommand*{\plogo}{\fbox{.}}



\renewcommand{\contentsname}{\textbf{目录}}

\setcounter{secnumdepth}{0}


\usepackage{fancyhdr}

\renewcommand{\headrulewidth}{0.0pt}
\renewcommand{\footrulewidth}{0.0pt}

\fancyfoot[CO,RE]{}
\pagestyle{fancy}
\fancyhf{}
\fancyhead[ER]{\leftmark}
\fancyhead[OL]{\rightmark}
\fancyhead[EL,OR]{\thepage}
\renewcommand{\headrulewidth}{0.4pt}
\fancypagestyle{plain}{%
    \fancyhf{}
    \fancyfoot[C]{\bfseries \thepage}
    \renewcommand{\headrulewidth}{0pt}
    \renewcommand{\footrulewidth}{0pt}
}
\newcommand{\Author}{\href{https://github.com/DingShizhe}{DingShizhe}}
\newcommand\myquotepage[3]{\thispagestyle{empty}\vspace*{\fill}\vspace*{\fill}\textcolor{#1}{\textit{#2\\\rightline{#3}}}\vspace*{\fill}\vspace*{\fill}\vspace*{\fill}}




\begin{document}


\setlength{\parindent}{0pt}
\setlength{\baselineskip}{20pt}



\begin{titlepage}
	\raggedleft
	\rule{1pt}{\textheight} % vertical line
	\hspace{0.05\textwidth}
	\parbox[b]{0.80\textwidth}{
		\scalebox{1.2}{\it\Huge\textit 拾穗}\\[2\baselineskip] % Title
		\textcolor{Sienna}{\large\textit{和光同尘}}\\[4\baselineskip]
		{\large\textsc{\Author}}
        
		\vspace{0.5\textheight}	
		\color{Sienna}{\noindent \textit 且慢~~\plogo}\\[\baselineskip]
	}
    
\end{titlepage}

%----------------------------------------------------------------------------------------


    \input{./build/句/1-年轻的时候,我想成为任何人,除了我自己。.tex}

\pagestyle{plain}

\tableofcontents

\setcounter{page}{1}

\newpage

\mainmatter

%\pagestyle{fancy} 

\chapter{诗}
\newpage

    \input{./build/句/2-蒹葭苍苍,白露为霜。所谓伊人,在水一方。.tex}
    
    \input{./build/诗/佚名-涉江采芙蓉.tex}
    \input{./build/诗/王维-送别.tex}
    \input{./build/诗/李白-春思.tex}
    \input{./build/诗/李商隐-晚晴.tex}
    
    \input{./build/句/3-生年不满百,常怀千岁忧。昼短苦夜长,何不秉烛游?.tex}
    
    \input{./build/句/4-若士必怒,伏尸二人,流血五步,天下缟素。.tex}
    
    \input{./build/诗/王绩-野望.tex}
    \input{./build/诗/王勃-送杜少府之任蜀州.tex}
    \input{./build/诗/杜甫-望岳.tex}

    \input{./build/诗/李商隐-乐游原.tex}
    \input{./build/句/9-人生天地间,忽如远行客。.tex}
    \input{./build/诗/杜甫-旅夜书怀.tex}
    \input{./build/诗/杜甫-春夜喜雨.tex}

    \input{./build/句/5-岁寒,然后知松柏之后凋也。.tex}
    
    \input{./build/句/7-人生不相见,动如参与商。今夕复何夕,共此灯烛光。.tex}
    

    \input{./build/诗/王维-杂诗三首其二.tex}

    \input{./build/诗/张九龄-望月怀远.tex}

    \input{./build/诗/张九龄-感遇其一.tex}

    \input{./build/诗/王维-红豆.tex}

    \input{./build/句/10-我们自古以来,就有埋头苦干的人,有拼命硬干的人.tex}

    \input{./build/句/11-在秋风鲈鱼之前,尚有薄酒一樽。.tex}

    \input{./build/诗/宋之问-渡汉江.tex}
    \input{./build/诗/白居易-问刘十九.tex}
    
    \input{./build/句/12-即便我身处果壳之中,仍自以为是无限宇宙之王。.tex}

    \input{./build/句/13-长太息以掩涕兮,哀民生之多艰!.tex}

    \input{./build/诗/李绅-悯农二首.tex}

    \input{./build/句/25-每个人对于他所属于的社会都负有责任,那个社会的弊他也有一份。.tex}
    \input{./build/诗/张俞-蚕妇.tex}

    \input{./build/句/17-我一直很偏执地记录这些人,甚至到了他们自己都忘记自己的时候。.tex}

    \input{./build/诗/梅尧臣-陶者.tex}
    \input{./build/诗/佚名-氓.tex}


    \input{./build/句/14-滚滚长江东逝水,浪花淘尽英雄。.tex}
    % \input{./build/诗/王健-历史的天空.tex}
    \input{./build/句/20-湮没了黄尘古道,荒芜了烽火边城。.tex}

    \input{./build/句/16-有匪君子,如切如磋,如琢如磨。.tex}
    

    \input{./build/诗/佚名-苕之华.tex}

    \input{./build/诗/李白-赠从兄襄阳少府皓.tex}

    \input{./build/句/19-子在川上曰:逝者如斯夫。.tex}

    \input{./build/诗/李白-梦游天姥吟留别.tex}

    \input{./build/句/21-三军可夺帅也,匹夫不可夺志也。.tex}

    \input{./build/诗/王维-竹林馆.tex}

    \input{./build/诗/黄庭坚-寄黄几复.tex}

    \input{./build/诗/李白-渡荆门送别.tex}


\chapter{词}
    \newpage
    \input{./build/句/8-一弹戏牡丹,一挥万重山。.tex}
    
    \input{./build/诗/苏轼-浣溪沙·细雨斜风作晓寒.tex}
    \input{./build/诗/苏轼-念奴娇·赤壁怀古.tex}
    \input{./build/诗/苏轼-蝶恋花·春景.tex}
    \input{./build/诗/秦观-鹊桥仙·纤云弄巧.tex}
    \input{./build/诗/柳永-雨霖铃.tex}
    \input{./build/诗/周邦彦-少年游.tex}

    \input{./build/诗/柳永-八声甘州·对潇潇暮雨洒江天.tex}
    \input{./build/诗/辛弃疾-青玉案·元夕.tex}
    \input{./build/诗/纳兰性德-山一程·长相思.tex}
    \input{./build/诗/周邦彦-苏幕遮·燎沉香.tex}
    \input{./build/诗/苏轼-望江南·超然台作.tex}
    \input{./build/诗/晏几道-临江仙·梦后楼台高锁.tex}
    \input{./build/诗/李白-忆秦娥·箫声咽.tex}

\chapter{现代诗}
    \newpage
    \input{./build/句/15-我和世界皆不完美。.tex}
    \input{./build/诗/毕赣-2016.1.5.tex}

    \input{./build/诗/余秀华-风从田野上吹过.tex}
    \input{./build/诗/张枣-镜中.tex}
    \input{./build/诗/余秀华-写给门前的一棵树.tex}

    \input{./build/诗/李叔同-送别.tex}
    \input{./build/句/6-问余何适,廓尔忘言,华枝春满,天心月圆。.tex}
    \input{./build/诗/北岛-真的.tex}
    \input{./build/诗/余秀华-穿越大半个中国去睡你.tex}
    \input{./build/诗/郑愁予-错误.tex}
    \input{./build/诗/余秀华-我爱你.tex}
    \input{./build/诗/废名-十二月十九夜.tex}

    \input{./build/句/23-酒入愁肠,七分酿成月光。余下的三分呼为剑气,绣口一吐便是半个盛唐。.tex}

    \input{./build/诗/余光中-飞将军.tex}

\chapter{文}

    \setlength{\parindent}{15pt}

    \newpage

    \input{./build/句/22-臣本布衣,躬耕于南阳。.tex}

    \input{./build/文/吴均-与朱元思书.tex}

    \input{./build/句/26-天下兴亡,匹夫有责。.tex}

    \input{./build/文/萧铎-采莲赋.tex}

    \input{./build/文/曹植-洛神赋.tex}

    \input{./build/文/李白-春夜宴从弟桃花园序.tex}
    \input{./build/文/周敦颐-爱莲说.tex}

    \input{./build/句/20-人固有一死,或重于泰山,或轻于鸿毛。.tex}

    \input{./build/句/18-孔子曰:何陋之有?.tex}

    \input{./build/句/24-燕雀安知鸿鹄之志哉!.tex}

    \input{./build/文/苏轼-前赤壁赋.tex}

\end{document}
