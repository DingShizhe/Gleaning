\documentclass[a5paper, 12pt]{book}

\usepackage{xeCJK, geometry, fancyhdr, titlesec}

\usepackage[
    CJKbookmarks = true,
    bookmarks = true,
    linktocpage = true,
    hyperindex = true,
    bookmarks = true
]{hyperref}

\hypersetup{%
    pdfborderstyle={/S/U/W 0.5}% border style will be underline of width 1pt
}


\usepackage[svgnames]{xcolor}

\geometry {
    a5paper,
    scale = 0.8,
    left = 2cm,
    right = 2cm,
    top = 2.5cm,
    bottom = 2.8cm
}

\newcommand{\htoday}{
    \href{https://github.com/DingShizhe}{\today}
    }

\usepackage{draftwatermark}
\SetWatermarkText{\htoday}
\SetWatermarkLightness{0.97}

\fancyfoot[CO,RE]{}

\newcommand*{\plogo}{\fbox{$\mathcal{DS}$}}
\renewcommand{\headrulewidth}{0.4pt}
\renewcommand{\footrulewidth}{0.4pt}

\renewcommand{\contentsname}{\TitleKai{目录}}

\setcounter{secnumdepth}{0}

\xeCJKDeclareSubCJKBlock{Punct}{
    `,,
    `。,
    `?,
    `—,
    `!,
    `、,
    `;,
    `:,
    `《,
    `》,
    `〈,
    `〉,
    `…,
    `~,
    `「,
    `」,
    `『,
    `』,
    `〔,
    `〕,
    `“,
    `”,
    `‘,
    `’
}

\setCJKmainfont[
    BoldFont={FandolHei},
   % BoldFont=方正大标宋_GBK,
   % BoldFont=Microsoft YaHei,
   % BoldFont=方正黑体_GBK,
   % BoldFont=方正报宋_GBK,
   Punct=方正楷体_GBK
% ]{方正楷体_GBK}
% ]{DFKai-SB}
]{HYKaiTiS}

\setCJKfamilyfont{TitleKai}{FandolKai}
\newcommand{\TitleKai}{\CJKfamily{TitleKai}}

\pagestyle{fancy}
\fancyhf{}
\fancyhead[ER]{\leftmark}
\fancyhead[OL]{\rightmark}
\fancyhead[EL,OR]{\thepage}
\renewcommand{\headrulewidth}{0.4pt}
\fancypagestyle{plain}{%
    \fancyhf{}
    \fancyfoot[C]{\bfseries \thepage}
    \renewcommand{\headrulewidth}{0pt}
    \renewcommand{\footrulewidth}{0pt}
} 
\newcommand{\Xiang}{\href{https://github.com/DingShizhe}{Xiang}}
\newcommand\myquotepage[3]{\thispagestyle{empty}\vspace*{\fill}\vspace*{\fill}\textcolor{#1}{\textit{#2\\\rightline{#3}}}\vspace*{\fill}\vspace*{\fill}\vspace*{\fill}}


\begin{document}
    \setlength{\parindent}{0pt}
    \setlength{\baselineskip}{20pt}

\begin{titlepage}
	\raggedleft
	\rule{1pt}{\textheight} % vertical line
	\hspace{0.05\textwidth}
	\parbox[b]{0.80\textwidth}{
		\scalebox{1.2}{\it\Huge\TitleKai 拾穗}\\[2\baselineskip] % Title
		\textcolor{Sienna}{\large\TitleKai{和光同尘}}\\[4\baselineskip]
		{\Large\textsc{\Xiang}}
        
		\vspace{0.5\textheight}	
		\color{Sienna}{\noindent \TitleKai 且慢~~\plogo}\\[\baselineskip]
	}
    
\end{titlepage}

%----------------------------------------------------------------------------------------


    \input{./build/句/1-年轻的时候,我想成为任何人,除了我自己。.tex}

\pagestyle{plain} 

\tableofcontents

\setcounter{page}{1}

\newpage

\mainmatter

\pagestyle{fancy} 

\chapter{诗}
\newpage
    
    \input{./build/句/2-蒹葭苍苍,白露为霜。所谓伊人,在水一方。.tex}
    
    \input{./build/诗/涉江采芙蓉-佚名.tex}
    \input{./build/诗/送别-王维.tex}
    \input{./build/诗/春思-李白.tex}
    \input{./build/诗/晚晴-李商隐.tex}
    
    \input{./build/句/3-生年不满百,常怀千岁忧。昼短苦夜长,何不秉烛游?.tex}
    
    \input{./build/句/4-若士必怒,伏尸二人,流血五步,天下缟素。.tex}
    
    \input{./build/诗/野望-王绩.tex}
    \input{./build/诗/送杜少府之任蜀州-王勃.tex}
    \input{./build/诗/望岳-杜甫.tex}

    \input{./build/诗/乐游原-李商隐.tex}
    \input{./build/句/9-人生天地间,忽如远行客。.tex}
    \input{./build/诗/旅夜书怀-杜甫.tex}
    \input{./build/诗/春夜喜雨-杜甫.tex}

    \input{./build/句/5-岁寒,然后知松柏之后凋也。.tex}
    
    \input{./build/句/7-人生不相见,动如参与商。今夕复何夕,共此灯烛光。.tex}
    

    \input{./build/诗/杂诗三首其二-王维.tex}

    \input{./build/诗/望月怀远-张九龄.tex}

    \input{./build/诗/感遇其一-张九龄.tex}

    \input{./build/诗/红豆-王维.tex}


    \input{./build/句/10-我们自古以来,就有埋头苦干的人,有拼命硬干的人.tex}

    \input{./build/句/11-在秋风鲈鱼之前,尚有薄酒一樽。.tex}

    \input{./build/诗/渡汉江-宋之问.tex}
    \input{./build/诗/问刘十九-白居易.tex}
    
    \input{./build/句/12-即便我身处果壳之中,仍自以为是无限宇宙之王。.tex}

    \input{./build/句/13-长太息以掩涕兮,哀民生之多艰!.tex}

    \input{./build/诗/悯农二首-李绅.tex}
    \input{./build/诗/蚕妇-张俞.tex}

    \input{./build/句/17-我一直很偏执地记录这些人,甚至到了他们自己都忘记自己的时候。.tex}

    \input{./build/诗/陶者-梅尧臣.tex}
    \input{./build/诗/氓-佚名-诗经.tex}


    \input{./build/句/14-滚滚长江东逝水,浪花淘尽英雄。.tex}
    % \input{./build/诗/历史的天空-王健.tex}
    \input{./build/句/20-湮没了黄尘古道,荒芜了烽火边城。.tex}

    \input{./build/句/16-有匪君子,如切如磋,如琢如磨。.tex}
    

    \input{./build/诗/苕之华-佚名-诗经.tex}

    \input{./build/诗/赠从兄襄阳少府皓-李白.tex}

    \input{./build/句/19-子在川上曰:逝者如斯夫。.tex}


    \input{./build/诗/梦游天姥吟留别-李白.tex}

    \input{./build/句/21-三军可夺帅也,匹夫不可夺志也。.tex}


\chapter{词}
    \newpage
    \input{./build/句/8-一弹戏牡丹,一挥万重山。.tex}
    
    \input{./build/诗/浣溪沙·细雨斜风作晓寒-苏轼.tex}
    \input{./build/诗/念奴娇·赤壁怀古-苏轼.tex}
    \input{./build/诗/蝶恋花·春景-苏轼.tex}
    \input{./build/诗/鹊桥仙·纤云弄巧-秦观.tex}
    \input{./build/诗/雨霖铃-柳永.tex}
    \input{./build/诗/少年游-周邦彦.tex}

    \input{./build/诗/八声甘州·对潇潇暮雨洒江天-柳永.tex}
    \input{./build/诗/青玉案·元夕-辛弃疾.tex}
    \input{./build/诗/山一程·长相思-纳兰性德.tex}
    \input{./build/诗/苏幕遮·燎沉香-周邦彦.tex}
    \input{./build/诗/望江南·超然台作-苏轼.tex}
    \input{./build/诗/临江仙·梦后楼台高锁-晏几道.tex}

\chapter{现代诗}
    \newpage
    \input{./build/句/15-我和世界皆不完美。.tex}
    \input{./build/诗/2016.1.5-毕赣.tex}

    \input{./build/诗/风从田野上吹过-余秀华.tex}
    \input{./build/诗/镜中-张枣.tex}
    \input{./build/诗/写给门前的一棵树-余秀华.tex}

    \input{./build/诗/送别-李叔同.tex}
    \input{./build/句/6-问余何适,廓尔忘言,华枝春满,天心月圆。.tex}
    \input{./build/诗/真的-北岛.tex}
    \input{./build/诗/穿越大半个中国去睡你-余秀华.tex}
    \input{./build/诗/错误-郑愁予.tex}
    \input{./build/诗/我爱你-余秀华.tex}
    \input{./build/诗/十二月十九夜-废名.tex}

    \input{./build/句/23-酒入愁肠,七分酿成月光。余下的三分呼为剑气,绣口一吐便是半个盛唐。.tex}

    \input{./build/诗/飞将军-余光中.tex}

\chapter{文}
    \newpage

    \input{./build/句/22-臣本布衣,躬耕于南阳。.tex}

    \input{./build/诗/与朱元思书-吴均.tex}

    \input{./build/诗/采莲赋-萧铎.tex}

    \input{./build/诗/洛神赋-曹植.tex}

    \input{./build/诗/春夜宴从弟桃花园序-李白.tex}
    \input{./build/诗/爱莲说-周敦颐.tex}

    \input{./build/句/20-人固有一死,或重于泰山,或轻于鸿毛。.tex}

    \input{./build/句/18-孔子曰:何陋之有?.tex}

\end{document}
